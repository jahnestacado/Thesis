\chapter{Longest Common Subsequence}
While trying to find a way to reduce false positives of the doc and xls fragment classification, we thought to test the performance and accuracy of the longest common subsequence technique. Calhoun[] used this technique to distinguish between fragments of two different file types. He achieved high accuracy results(~90\%) using the standard dynamic programming version of the algorithm. Even though the dynamic version is faster than the naive approach of the algorithm, with runtime complexity $mxn$, where $m n $ the length of the input strings, it still seems like an "expensive" technique to be used in file carving. He extracted the longest common subsequences of every file fragment in his training set and concatenated them in a big string. This string is used as a representative of the respective file type.  Due to the fact that the speed of this technique depends on the length of the input strings, it is essential to know about how long the file type representative string should be in order to be effective. Since he does not provide information about the length of the strings that he used as file type representatives , we want to find out strings of what length can be used as file type representatives and yield similar results. If the lengths are not too long then the computation of the longest common subsequence between two strings could be fast enough to be used in file carving techniques.\\\\
 Instead of concatenating every longest common subsequence between fragments of the same file type, we tried a different approach. We used 500 fragments of the doc and xls type for our representative string creation. This resulted to $500x500 - 500 = 249,500$ comparisons for each file type. We gathered all longest common subsequences from these comparisons and putted them in a map data structure. Then we sorted the map and took the first 100, 500, 1000 and 1500 most frequent longest common subsequences. We concatenated these subsequences in 4 long representative strings for each of the doc and xls file type. Thereafter, we used a set of 10,000 fragments, 5000 of xls and 5000 of doc type, to test their accuracy. At this point we should note that Calhoun used only 50 fragments per file type to test this technique. This was an additional reason to want to try its performance since we consider testing sets of this size extremely insufficient. However, we also consider our testing set significantly small, but since our goal was to test the correlation between the speed and the accuracy of that technique, that size is acceptable. The results can be found in Table \ref{table:lcs}.\\\\
%%------------------------------------------------------------------------------------------------------------------------------
\begin{table}
\centering
\caption{Longest Common Subsequence comparison for doc vs. xls\label{table:lcs}}
\colorbox{blue!30}{
\scalebox{0.75}{

\begin{tabular}{ l c c c c c c c c c c }
\hline
\hline
& $n$ most frequent lcs          & $n=100$	 & $n=500$   & $n=1000$  & $n=1500$       \\[0.6ex]
\hline
\\[0.2ex]
doc vs. xls precision 		        & & 83        & 89.5      & 90.06     & 91.63 \\\\[0.8ex]
doc lcs representative string length & & 1,007      & 5,763    & 15,225    & 27,070     \\[0.8ex]
xls lcs representative string length & & 859       & 4,679     & 9,482     & 14,609    \\[0.8ex]


\end{tabular}}}
\end{table}
%End lcs table
As someone would expect, using longer strings as file type representatives result in higher classification precision. The precision gradually increases while using longer strings. However, Although the precision of this metric proved to be in pair with the results Calhoun presented [], its speed is way to slow to be used in real life cases. Even by using the shortest file type representative strings, which corresponds to the first 100 most frequent longest common subsequences of a file type, the runtime complexity remains extremely high. We compared the speed of this technique with our unoptimized algorithms speed, and although our benchmarking is not completely accurate, the longest common subsequent technique takes 56\% more time to compute than our complete algorithm(BFA included). Taking under account that our algorithm was designed to be able to handle 10 different file types and that the LCS technique that we tried only 2, it is obvious that the difference in speed is quite significant. Moreover, since our algorithm yielded few false positives for xls fragment classification, we don't think that we could improve our overall accuracy by using the LCS technique. In conclusion, we strongly believe that such an expensive technique is not appropriate for broad fragment classification and researchers should first invest time in searching for light weight techniques before trying brute force approaches. 
 
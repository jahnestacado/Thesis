\chapter{Results}\label{chap:6}
In this chapter we are going to present the accuracy results of our classification algorithm. Analysis of the results will be presented in the next chapter.
Since our algorithms design is based on the analysis we did on the experimental data set(Table ~\ref{table:data set}), the algorithm is biased towards this set. For that reason we used the final testing data set(Table X) for testing our algorithms performance.

Although our final algorithm was implemented in one piece, we divide the results in two parts. In section ~\ref{sec:6.1} we  provide the performance of the first half of our algorithm, which is the BFA, regarding text fragment classification. In section ~\ref{sec:6.2} we provide the final classification results of our complete algorithm. We should note that the final classification results correspond to the data set that is comrpised of fragments that were initially classified as text from the BFA.
\pagebreak

\section{BFA Scan - Text Fragment Classification}\label{sec:6.1}
In Table \ref{table:final_bfa} we present the performance of the first part of our algorithm regarding text fragment classification. The first row corresponds to the initial number of fragments our algorithm processed for each file type. The second and the third row provide information regarding the amount of fragments that were classified as text from BFA.
%%------------------------------------------------------------------------------------------------------------------------------
\begin{table}[H]
\centering
\caption{BFA Text Fragments Classification\label{table:final_bfa}}
\colorbox{blue!25}{
\scalebox{0.65}{

\begin{tabular}{ l c c c c c c c c c c }
\hline
\hline
 & pdf & text & doc & xls & ppt  & mp4 & ogg & zip & png & jpg  \\ 
\hline
num. of fragments  & 1.874.910 & 1.891.472 & 1.775.747 & 1.870.376 & 1.864.145 & 1.888.605  & 1.891.754 & 1.889.477 & 1.880.742 & 1.886.853 \\[0.6ex]
\hline
\\[0.2ex]
fragments classified as text & 385.616 & 1.889.662  & 950.308 & 1.756.200 & 404.198 & 67.107 & 111.738 & 94.815 & 86.017 & 98.960     \\[0.6ex]
 \multicolumn{1}{r}{\%}   & 20,6  & 99,9  & 53,5 & 93,9 & 21,7 & 3,6 & 5,9 & 5 & 4,6 & 5,2     \\[0.6ex]
 
 fragments classified as other & 1.489.294  &1.810  & 825.439 & 114.176 & 1.459.947 & 1.821.498  & 1.780.016  & 1.794.662  & 1.794.725  & 1.787.893 \\[0.6ex]
 \multicolumn{1}{r}{\%}   & 79,4 & 0,1  & 46,5  &  6,1 & 78,3 & 96,4  & 94,1  & 95 & 95,4  & 94,8     \\\\[0.6ex]



\end{tabular}}}
\end{table}
%End sample table




\section{Algorithm Accuracy}\label{sec:6.2}
In Table \ref{table:final} we provide the final accuracy results of our algorithm as a percentage confusion matrix. The columns represent the actual type of the fragments while the rows represent the file type that the fragments was classified as. Since our algorithm was designed to classify fragments of the doc, pdf, xls and text file formats, we only have true positives percentages for these 4 file types. We present the true positive values as shaded cells.

%%------------------------------------------------------------------------------------------------------------------------------
\begin{table}[H]
\centering
\caption{BFA Extension Algorithm Accuracy\label{table:final}}
\colorbox{blue!25}{
\scalebox{0.65}{

\begin{tabular}{ l c c c c c c c c c c }
\hline
\hline
 & pdf & text & doc & xls & ppt  & mp4 & ogg & zip & png & jpg  \\ 
\hline
fragments classified as text from BFA & 385,616 & 1,889,662  & 950,308 & 1,756,200 & 404,198 & 67,107 & 111,738 & 94,815 & 86,017 & 98,960     \\[0.6ex]
\hline
\\[0.2ex]
pdf 		 &\cellcolor{blue!15} 46.3  & 0.5  & 16.2  & 2.3   & 31.2  & 87.6  & 95.5 & 90.6  & 84.9& 83.0    \\[0.6ex]
text		 & 38.9  &\cellcolor{blue!15} 98.8    & 11.7  & 3.9   & 1.2   & 0     & 0.1   & 0  & 2.2 & 6.4\\[0.6ex]
doc 		 & 13.6  & 0.7   &\cellcolor{blue!15} 60.7  & 17.9  & 43.4  & 12.4  & 4.4  & 9.3 & 8.8 & 8.7   \\[0.6ex]
xls 		 & 1.3   & 0     & 11.4  &\cellcolor{blue!15} 75.9  & 24.2  & 0.1   & 0    & 0.2    & 4.1 & 2 \\[0.6ex]


\end{tabular}}}
\end{table}
%End sample table
\chapter{Discussion}
In our experiment, during our algorithm development procedure our analysis yielded two new classification metrics. The Individual Null Byte Frequency(INBF) and the Plain Text Concentration(PTC). The representative INBF values that we used to classify mainly document-type fragments were formed from BFAs output. The output was comprised only of fragments that were classified as text from BFA. This resulted to an unpropitious set of fragments of the 4 document-type file formats. The amount of pdf and doc fragments were significantly less than the amount of text and xls. Although INBF seems to be quite effective as a part of our classification algorithm, we believe that a more extensive analysis must be made in a bigger data set, in order to be able to say if this metric could provide aid for broad fragment classification. On the other hand, PTC analysis was made in a corpus of 10GB in total and we believe that our analysis is consistent and can be a great asset in file carving techniques.

Although we did our best to eliminate possible biases in our experimental setup, we can not guarantee the integrity of our corpus. Considering that our corpus was comprised of tens of thousands of files it wasn't feasible to manually check if the suffixes of the files correspond to the actual file type. We did some manual inspections in the experimental data set and we found about 30mbs of files that had a $.txt$ suffix but weren't text files. Additionally, we don't know if our corpus was comprised of various file format versions. For example, an Adobe PDF 1.7 document might be slightly different from an Adobe PDF 1.3 document.
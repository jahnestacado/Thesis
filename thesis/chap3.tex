\chapter{Experimental Setup}


\section{Data Set}
The data set we used for both our training and testing procedures is derived from  Garfinkels[] coprus, Wikipedia and Academic Earth[] and is the exact same coprus that Shahi[] used for his testing set.The set is comprised of 10 different file types with a size of about 1GB each. We split this corpus in a 9-1 ratio for the training and testing set respectively. 
Furthermore, we divided both the testing and training set files in 512-byte blocks, which we refer to them as fragments. We used the training set to train our fingerprints and apply statistical analysis in order to discover useful patterns and the testing set to test all variations of our BFA algorithm. More detailed information about our data set can be found in Table ~\ref{table:data set}.
%%------------------------------------------------------------------------------------------------------------------------------
\begin{table}
\centering
\caption{Experimental Data Set\label{table:data set}}
\colorbox{blue!30}{
\scalebox{0.62}{

\begin{tabular}{ l r r r r r r r r r r }
\hline
\hline
 & pdf  &  zip  &  text  &  doc  &  mp4  &  xls  &  ppt  &  jpg  &  ogg  &  png \\ [0.6ex]
\hline\\
\uline{Training Set} \\[0.6ex]
num.of files                               & 1,642    & 1        & 954      & 1,697    & 1        & 373      & 193      & 1,781    &464   & 4,395    \\[0.6ex]
size in megabytes                          & 869.3	 & 860.6    & 831.2    & 867.6    & 813.6    & 869.5    & 866.9    & 870.5    &863.4 & 868.9 \\[0.6ex]
expected num. of fragments                 & 1,780,326& 1,762,508& 1,702,297& 1,776,844& 1,666,252& 1,780,736& 1,775,411& 1,782,784&1,768,243 &1,779,507    \\[0.6ex]
output num.of fragments                    & 1,694,034& 1,680,771& 1,622,534& 1,467,314& 1,588,908& 1,684,374& 1,683,444& 1,698,877&1,685,954 &1,692,813    \\[0.6ex]  
percentage of fragments with no plain text & 4.8	     & 4.6      & 4.7      & 17.4     & 4.6      & 5.4      & 5.2      & 4.7      &4.7   &4.9  \\\\\\

\uline{Testing Set} \\[0.6ex]
num.of files                               & 217	   & 1      & 367    & 257    & 1      & 81     & 35     & 214    & 101    & 555    \\[0.6ex]
size in megabytes                          & 100	   & 104.9  & 97.4   & 100.2  & 104.9  & 100.2  & 100.6  & 100.2  & 100.2  & 101.5    \\[0.6ex]
expected num. of fragments                 & 204,800& 214,835& 199,475& 205,209& 214,835& 205,209& 206,028& 205,209& 205,209& 207,872    \\[0.6ex]
output num.of fragments                    & 189,732& 204,795& 190,055& 177,887& 204,728& 193,352& 195,289& 195,608& 195,656& 195,653   \\[0.6ex]
percentage of fragments with no plain text & 7.4    & 4.7    & 4.7    & 13.3   & 4.7    & 5.8    & 5.2    & 4.7    & 4.7    & 5.9 \\

\end{tabular}}}
\end{table}



\section{Byte Frequency Analysis(BFA) Algorithm}
BFA [McDaniels] is a statistical learning algorithm that was initially developed to analyse and classify whole files. It was not meant to be used in file fragments. By counting the number of instances of each byte in a file of a certain type, BFA uses this frequencies to create a representative average value for each byte instance, among with their respective correlation strength. This results in a fingerprint for this particular file-type. Then during the classification procedure, the input file is compared with every fingerprint and an accuracy level is created for each file type. BFA classifies the file to the file type that corresponds to the highest accuracy level.
 Shahi trained and tested BFA with file fragments of 512-byte size and his results show that although the algorithm is pretty bad for broad fragment classification, it is quite good in identifying fragments that belong to document-type files as text. We use a BFA which will train our fingerprints with the bytes that corresponds to the printable ASCII characters plus the tab, line break and carriage return instead of the complete byte-set of the fragments. This BFA will be only the first step of our algorithm and we intend to use additional metrics after this point. Taking under account the speed requirements, BFA seems as a very good candidate since it is quite a lightweight technique compared to heavier machine learning algorithms. Moreover, as we already stated, BFA seems to classify a big ammount of document-type fragments as text. Shahi tested several classification algorithms in the same corpus. BFA was also tested among with Byte Frequency Correlation algorithm, n-Gram Analysis and Conti et al. algorithm. The results show that BFA has the highest precision in classifying document-type fragments as text.




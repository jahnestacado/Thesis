% Chapter Template

\chapter{Case Study} % Main chapter title

\label{Chapter5} % Change X to a consecutive number; for referencing this chapter elsewhere, use \ref{ChapterX}

\lhead{Chapter 5. \emph{Case Study}} % Change X to a consecutive number; this is for the header on each page - perhaps a shortened title

%----------------------------------------------------------------------------------------
%	SECTION 1
%----------------------------------------------------------------------------------------
In this chapter we present the evaluation we did. In order to evaluate our study and prove that we can use the interface we created for RASCAL to perform analyses with MS Office Documents, we will try to replicate one study conducted by Felienne Hermans et all \cite{feliene1}. The study is focusing on extracting facts from spreadsheets formulas by analysing them, and gaining relevant knowledge about common errors-smells .

\section{Detecting Code Smells in Spreadsheet Formulas}
In this study, they tried to identify a list of metrics with which we can detect formula smells in relation to code smells as defined in Software Engineering area. They identified five smells and then they created by extracting facts from EUSES Spreadsheet corpus\cite{eusesCorpus} a threshold map for these five metrics. Each metric has three thresholds, each one for the categories of low, medium and high risk. Later they conducted an experiment to find the affected files by these smells in the EUSES Spreadsheet Corpus where the results can be shown in table \ref{tab:eusesRes}. Also, they conducted a second experiment where they asked from users to identify formula smells in their spreadsheets. Their results from the evaluations have shown that smells can reveal probably error prone formulas.

The five metrics that are used in this study are quite related to the code smells known in software engineering area. They have identified the following smells :

\begin{description}
\item [Multiple Operations] Like the long method smell in programming, the Multiple Operations smell identifies formulas that contain a lot of operations. This introduces a threat to readability. For example A4+B2/(C5*D3\^Y2) contains multiple operations that can be done separately, because the user will have to trace many cells in order to understand the semantics of this complex formula.
\item[Multiple References] This smell is related to the many parameters smell in programming. A method that uses many inputs as values can be split in more cells to enhance readability. For example the formula SUM(A1:A6; C1; C2; C3; V4:V8). It's easily to understand that its the summary of all these cells, but in order to view the original values you have to trace to five different locations.
\item[Conditional Complexity] Conditional complexity stands for the same purpose as in programming language, since in spreadsheets we can define alternative flows using conditional statements. The higher the complexity in a formula is, the more difficult to understand the purpose of that formula.
\item[Long Calculation Chain] Formulas can contain references to other cells with formulas, which further can create a chain of calculations. Long Calculation Chain metric, calculates the higher number of steps you can do to find plain data. Although, this is in contradiction with the first two metrics. 
\item[Duplicated Formulas] In relation with duplicated code smell, duplicated formulas smell identify the similarity between two formulas. For example consider SUM(A1:A6)+10\% and SUM(A1:A6). These two formulas are containing the same sub-part and can be considered duplicated.
\end{description}

In order to determine thresholds for the metrics, so we can use them as smell indicators, they cleansed the EUSES Corpus to obtain only unique formulas, and they performed measurements to identify the threshold levels. In table \ref{tab:thres} we can see the threshold map for the five metrics. Each metric has three different thresholds with label 70, 80 and 90\% which are corresponding to risk levels low, moderate and high. For further information about the threshold identification process one can read the original study report\cite{feliene1}.

\begin{table}
\centering
    \begin{tabular}{llll}
    \hline
    Smell                  & 70\% & 80\% & 90\% \\\hline
    Multiple Operations    & 4   & 5   & 9   \\
    Multiple References    & 3   & 4   & 6   \\
    Conditional Complexity & 2   & 3   & 4   \\
    Message Chain          & 4   & 5   & 7   \\
    Duplication            & 6   & 9   & 13  \\
    \end{tabular}
    \caption{Thresholds that indicate formula smells\cite{feliene1}.}\label{tab:thres}
\end{table}

With the provided thresholds they conducted an experiment to measure the percentage of files in EUSES Spreadsheet Corpus, that suffer from the given smells. In table \ref{tab:eusesRes} we see the results from the conducted study.

\begin{table}
\centering
    \begin{tabular}{llll}
    \hline
    Smell                  & >70\%  & >80\%  & >90\% \\\hline
    Multiple Operations    & 21.6\% & 17.1\% & 6.3\% \\ 
    Multiple References    & 23.8\% & 18.4\% & 6.3\% \\
    Conditional Complexity & 4.4\%  & 3.0\%  & 1.1\% \\
    Message Chain          & 9.0\%  & 7.9\%  & 3.3\% \\
    Duplication            & 10.8\% & 10.8\% & 3.7\% \\
    \end{tabular}
    \caption{Percentage of EUSES spreadsheets that suffer from at least one of the five formula smells\cite{feliene1}.}\label{tab:eusesRes}
\end{table}

\section{Replication in Rascal}
In order to validate our results and to prove that our tool can be used to analyse Spreadsheets we replicated the before mentioned study, using the tool in RASCAL which we created and interfaced with Apache POI through our research. The generated ADT for Excel Spreadsheets can be found in Appendix \ref{AppendixB}.  We obtained the EUSES spreadsheet corpus and we implemented a module that simulates the process they used and runs the corresponding analyses.  

We covered three of the five metrics in this analysis to reduce time and effort since our goal is not to prove the validity of the original research but to prove that our tool can be used for this kind of analyses. Since Apache POI returns formulas as strings, we used pattern matching techniques to declare the functionality we wanted. Although, since RASCAL allows to declare grammars, one can declare the grammar of a formula and use RASCAL tools to parse the formula strings and obtain more accurate results. This is also out of the scope of this research so we used pattern matching techniques to perform the replications.

We declared Multiple Operations smell as the number of operations that we show in table \ref{tab:mulSmells}. For the Multiple References smell we defined the value as the number of references to cells contained in a formula. References can point to single cells directly, and to cell ranges, for example A4 and A5:A8 correspondingly. Also, in the report of the study we found out that there is no clear indication of what kind of references they have counted. The third smell we investigated is the Conditional Complexity smell where we defined it as the number of IF contained in the formula string. This means that we also encountered all the spreadsheet functions that ends with IF, for example SUMIF.

\begin{table}[h]
\centering
    \begin{tabular}{l}
    \hline
    Operators           \\
    \hline
    +  -  *  /             \\
    <  <=  >  >=  =  <> \\
      ! \^  \&            \\
      Excel functions: i.e. sum()
    \end{tabular}
    \caption{Operators we counted for the Multiple Operations smell.}\label{tab:mulSmells}
\end{table}

We used the EUSES Spreadsheet Corpus to run the analysis which we obtained from the EUSES Consortium and it contains 5,638 spreadsheets. The corpus contains different collections of spreadsheets, which is further divided in folders bad, duplicates and processed. The experiment run only on the files that were in the folder named \textsl{processed} as they did in the original study. The total sum of the files in these folders is 4512 spreadsheets, which means that our corpus is slightly different from the one in the original study. Although, some files could not be opened by Apache POI due to errors of the Apache POI itself. We present in table \ref{tab:poiErrors} the various reasons of the errors we encountered during this phase. These files were excluded from our set and we finally conducted the experiment with 4146 spreadsheets.

%present the code, and have in appendix the HSSFInterface
%say something about that we checked the errors and bla bla bla.we know its cause is due to apache poi innabilities.
A threat to the validity of our replication results is the usage of regular patterns instead of declaring a grammar for spreadsheet functions. Another threat might be the excluded files of the corpus that we could not process which can be seen in the Table \ref{tab:poiErrors}. Also, the difference in the total sum of spreadsheets contained in our corpus and in the corpus of the original study, may affect the results. Beyond these, our main purpose is to prove that our tool can provide the required functionality to perform this kind of analyses and not to validate the results of the original study.

\begin{table}
\centering
    \begin{tabular}{p{12.8cm}|c}
     \hline
    \textbf{File Importing Error}  & \textbf{Total}\\ \hline
    The supplied spreadsheet seems to be Excel 5.0/7.0 (BIFF5) format. POI only supports BIFF8 format (from Excel versions 97/2000/XP/2003)            bla       & 260 \\ \hline
    Invalid header signature; read 0x7C656C6946646162, expected 0xE11AB1A1E011CFD0                            & 49     \\\hline
    org.apache.poi.hssf.record.aggregates.CustomViewSettingsRecordAggregate cannot be cast to org.apache.poi.hssf.record.Record                    ~        & 4     \\\hline
    Unexpected record type (org.apache.poi.hssf.record.UncalcedRecord)  &1\\\hline
    Initialisation of record 0x868 left 8 bytes remaining still to be read.  & 4\\\hline
    Unable to read entire header; 87 bytes read; expected 512 bytes  & 6\\\hline
    Unexpected tAttr: org.apache.poi.ss.formula.ptg.AttrPtg []  & 2\\\hline
    Unable to construct record instance & 5\\\hline
    java.lang.ClassCastException  & 11\\\hline
    Cannot remove block[ -33554425 ]; out of range[ 0 - 1939 ]  & 1\\\hline
    java.lang.NumberFormatException & 1\\\hline
    Unexpected base token id (24)   &2 \\\hline
    13  & 1\\\hline
    -1  & 1\\\hline
    7  & 1\\\hline
    0  & 1\\\hline
    \end{tabular}
    \caption{File importing errors of Apache POI we encountered.}\label{tab:poiErrors}
\end{table}

\section{Results \& Conclusions}
In the table \ref{tab:ourRes} we see the results we obtained from the analysis we performed on our data set. We can observe in the results that in all metrics we achieved to obtain similar results with a deviation of 4\%.  By considering all the threats to validity that we referenced before, the difference between our results and the results of the original study is tolerable. 

\begin{table}[h]
\centering
    \begin{tabular}{llll}
    \hline
    Smell                  & >70\%  & >80\%  & >90\% \\\hline
    Multiple Operations    & 19.77\% & 16.16\% & 8.7\% \\ 
    Multiple References    & 22.28\% & 16.18\% & 9.76\% \\
    Conditional Complexity & 4.19\%  & 3.06\%  & 2.46\% \\
    \end{tabular}
    \caption{Results of our analysis in EUSES Spreadsheet Corpus.}\label{tab:ourRes}
\end{table}

The overall time the analysis run was 5356 seconds which can further be reduced. The ADT for HSSFWorkBook retrieves all the information of a spreadsheet using the Apache POI API. If we prune more the analysis tree with the way we presented in LABEL, then we can create a less complex structure which will use less calculations to obtain the data and create the corresponding ADT value in RASCAL, so we can reduce the required time for analysis.





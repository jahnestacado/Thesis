\bibitem{Aronson}      
Aronson, L., \& Bos, J. Van Den. (2011). Towards an Engineering Approach to File Carver Construction. 2011 IEEE 35th Annual Computer Software and Applications Conference Workshops, pp . 368–373.

\bibitem{MacDaniel}
M. McDaniel and M. Heydari , “ Content based file type detection algorithms ,” in Proc. 36th Annu. Hawaii Int. Conf. System Sciences (HICSS’03)—Track 9 , IEEE Computer Society , Washington , D.C. , 2003 , p. 332.1

\bibitem{Scalpel}
G. G. Richard , III and V. Roussev , “ Scalpel: A frugal, high performance file carver ,” in Proc. 2005 Digital Forensics Research Workshop (DFRWS) , New Orleans , LA , Aug. 2005 .

\bibitem{Pal}
Pal, A., \& Memon, N. (2009). The evolution of file carving. Signal Processing Magazine, IEEE, (March),pp. 59–71.

\bibitem{ngram}
Li W, Wang K, Stolfo S, Herzog B. Fileprints: identifying file types by n-gram analysis. In: IEEE information assurance workshop, 2005.

\bibitem{Garfinkel}
Garfinkel, S. L. (2007). Carving contiguous and fragmented files with fast object validation. Digital Investigation, 4, pp. 2–12

\bibitem{Sceadan}
Maddox, L., \& Beebe, N. (2012). Systematic Classification Engine and Data Analysis Overview by Dr . Nicole Beebe

\bibitem{Ashim}
Shahi, A. Classifying the classifiers for file fragment classification, August 2012.

\bibitem{Shannon}
Shannon CE. The mathematical theory of communication. Bell System Tech J 1948;27:379–423, 623–56.

\bibitem{roc}
M. Karresand and N. Shahmehri, "File Type Identification of Data Fragments by their Binary Structure," in Proceedings of the 7th Annual IEEE Information Assurance Workshop, 2006.

\bibitem{Veenman}
C. Veenman, "Statistical Disk Cluster Classification for File Carving," in Proceedings of the First Internation Workshop on Computational Forensics, 2007.

\bibitem{Calhoun}
Calhoun, W. C., \& Coles, D. (2008). Predicting the types of file fragments. Digital Investigation, 5, S14–S20.


\bibitem{Axelsson}
Axelsson, S. (2010). The Normalised Compression Distance as a file fragment classifier. Digital Investigation, 7, S24–S31. doi:10.1016/j.diin.2010.05.004

\bibitem{Gopal}
Gopal S, Yang Y, Salomatin K, Carbonell J. Statistical learning for file-type identification. In: 2011 10th International Conference on Machine Learning and Applications and Workshops (ICMLA), Vol. 1; 2011. p. 68–73.

\bibitem{Fitz}
Fitzgerald, S., Mathews, G., Morris, C., \& Zhulyn, O. (2012). Using NLP techniques for file fragment classification. Digital Investigation, 9, S44–S49.

\bibitem{Conti}
Conti G, Bratus S, Sangster B, Ragsdale R, Supan M, Lichtenberg A, et al. Automated mapping of large binary objects using primitive fragment type classification. In: Proceedings of the 2010 Digital Forensics Research Conference (DFRWS); 2010.

\bibitem{Garfinkel}
S. Garfinkel, P. Farrell, V. Roussev and G. Dinolt, "Bringing science to digital forensic with standardized forensic corpora," in DFRWS, 2009.

\bibitem{Corpora}
"Digital Corpora," [Online]. Available: http://digitalcorpora.org/.

\bibitem{A.Earth}
"Academic Earth," [Online]. Available: http://www.academicearth.org/.

\bibitem{Jeroen}
J. van den Bos and T. van der Storm, “Bringing Domain- Specific Languages to Digital Forensics,” in Proceedings of the 33rd ACM/IEEE International Conference on Software Engineering (ICSE’11), vol. 2. ACM, 2011.

\bibitem{Demsar}
Demsar, J. 2006. Statistical comparisons of classifiers over multiple data sets. Journal of Machine Learning Research 7:1-30.

\bibitem{Sokolova}
Sokolova, M., Japkowicz, N., \& Szpakowicz, S. (2006). Beyond accuracy, F-score and ROC: a family of discriminant measures for performance evaluation. AI 2006: Advances in Artificial Intelligence Lecture Notes in Computer Science Volume 4304, 2006, pp 1015-1021.